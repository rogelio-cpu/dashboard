\documentclass[12pt,a4paper]{article}
\usepackage[utf8]{inputenc}
\usepackage[T1]{fontenc}
\usepackage[french]{babel}
\usepackage{amsmath, amssymb}
\usepackage{graphicx}
\usepackage{geometry}
\usepackage{hyperref}
\usepackage{titlesec}
\usepackage{booktabs}
\usepackage{xcolor}
\usepackage{titling}
\usepackage{lipsum} % Pour le texte factice, à retirer ensuite

% Définition des couleurs
\definecolor{bleuprincipal}{RGB}{0, 102, 204}
\definecolor{grisclair}{RGB}{245, 245, 245}
\definecolor{bleufonce}{RGB}{0, 51, 102}
\definecolor{vertaccent}{RGB}{0, 153, 51}

% Configuration de la géométrie de la page
\geometry{margin=2.5cm}

% Suppression de l'indentation des paragraphes
\setlength{\parindent}{0pt}
\setlength{\parskip}{6pt}

% Personnalisation des sections avec couleurs
\titleformat{\section}
  {\normalfont\Large\bfseries\color{bleuprincipal}}
  {\thesection}{1em}{}

\titleformat{\subsection}
  {\normalfont\large\bfseries\color{bleufonce}}
  {\thesubsection}{1em}{}

\titleformat{\subsubsection}
  {\normalfont\normalsize\bfseries\color{vertaccent}}
  {\thesubsubsection}{1em}{}

% Personnalisation de la table des matières
\usepackage[titles]{tocloft}
\renewcommand{\cftsecfont}{\color{bleuprincipal}}
\renewcommand{\cftsubsecfont}{\color{bleufonce}}
\renewcommand{\cftsecpagefont}{\color{bleuprincipal}}
\renewcommand{\cftsubsecpagefont}{\color{bleufonce}}

% Configuration des liens hypertextes
\hypersetup{
    colorlinks=true,
    linkcolor=bleuprincipal,
    urlcolor=vertaccent,
    citecolor=bleufonce
}

% En-têtes et pieds de page personnalisés
\usepackage{fancyhdr}
\pagestyle{fancy}
\fancyhf{}
\fancyhead[L]{\textcolor{bleufonce}{\leftmark}}
\fancyhead[R]{\textcolor{bleufonce}{\thepage}}
\renewcommand{\headrulewidth}{0.5pt}
\renewcommand{\headrule}{\hbox to\headwidth{\color{bleuprincipal}\leaders\hrule height \headrulewidth\hfill}}

% Page de garde personnalisée
\renewcommand{\maketitle}{
  \begin{titlepage}
    \centering
    \vspace*{\fill}
    
    {\Huge\bfseries\color{bleuprincipal} Rapport Technique du Système AquaGuard\par}
    \vspace{0.5cm}
    {\Large\bfseries\color{bleufonce} Détection et Analyse des Fuites d'Eau en Temps Réel\par}
    \vspace{2cm}
    {\large\color{gray} Version 1.6\par}
    \vspace{0.5cm}
    {\large\color{gray} \today\par}
    
    \vspace*{\fill}
  \end{titlepage}
}

% Suppression du titre automatique
\title{}
\author{}
\date{}

\begin{document}

% Page de garde
\maketitle

% Table des matières avec couleurs
\tableofcontents
\thispagestyle{empty}
\newpage

% Première page avec style personnalisé
\pagestyle{fancy}
\setcounter{page}{1}

\section{Introduction}
AquaGuard est un système de surveillance métrologique conçu pour détecter les fuites d'eau avec une \textcolor{vertaccent}{précision optimale}. Il s'appuie sur une architecture distribuée comprenant un nœud IoT (Edge), un serveur de traitement backend (Node.js/Express) et une interface de visualisation en temps réel (Dashboard React).

\section{Architecture Matérielle et Acquisition}

\subsection{Composants}
Le nœud d'acquisition est articulé autour d'un microcontrôleur \textbf{\textcolor{bleufonce}{ESP32}}. La mesure est assurée par deux débitmètres à effet Hall (type YF-S201) positionnés en amont et en aval du réseau à surveiller.

\subsection{Acquisition des Données (Edge Computing)}
Le calcul du débit s'effectue par interruption (comptage d'impulsions). La formule de conversion intégrée dans le firmware est :
\begin{equation}
\boxed{Q \text{ (L/min)} = \frac{N_{\text{impulsions}}}{K} \times 60}
\end{equation}
Où $K = 450.0$ est la constante d'étalonnage. Les débits amont ($Q_{\text{in}}$) et aval ($Q_{\text{out}}$) sont mesurés chaque seconde et transmis au serveur via une requête HTTP POST sécurisée sur réseau Wi-Fi.

\section{Logique de Traitement et Filtrage (Backend)}
Afin de pallier l'imprécision inhérente aux capteurs (bruit de mesure) et statuer avec certitude sur la nature d'une fuite, le serveur applique un algorithme de traitement en plusieurs étapes.

\subsection{Calcul de l'Écart Différentiel}
À la réception des données en L/min, le système calcule l'écart différentiel brut et le convertit en ml/min :
\begin{equation}
\Delta Q \text{ (ml/min)} = (Q_{\text{in}} - Q_{\text{out}}) \times 1000
\end{equation}
Toute valeur négative est corrigée à $0$ par sécurité (le débit amont ne pouvant physiquement pas être inférieur au débit aval dans un circuit étanche).

\subsection{Filtrage et Seuils de Détection}
Pour éviter les faux positifs causés par les fluctuations hydrodynamiques (zone de silence), la règle métier suivante est strictement appliquée :

\textcolor{vertaccent}{\textbf{État Normal :}} Si $\Delta Q \le 300 \text{ ml/min}$. L'imprécision est gommée en forçant $\Delta Q = 0$ et le volume cumulé est remis à zéro.

\textcolor{bleufonce}{\textbf{Fuite Mineure :}} Si $300 < \Delta Q \le 800 \text{ ml/min}$.

\textcolor{bleuprincipal}{\textbf{Fuite Critique :}} Si $\Delta Q > 800 \text{ ml/min}$.

\subsection{Intégration Temporelle (Calcul du Volume)}
Le volume d'eau n'est pas une valeur statique mais le résultat de l'intégration mathématique du débit de fuite sur la durée de l'anomalie. Le serveur mémorise l'horodatage de la dernière trame ($t_{\text{prev}}$) et calcule la durée écoulée $\Delta t$ s'appliquant sur l'échantillon actuel :
\begin{equation}
\Delta t \text{ (s)} = t_{\text{actuel}} - t_{\text{prev}}
\end{equation}
Le volume cumulé perdu $V_{\text{perdu}}$ est incrémenté dynamiquement :
\begin{equation}
\boxed{V_{\text{perdu}} \text{ (ml)} = V_{\text{précédent}} + \frac{\Delta Q \times \Delta t}{60}}
\end{equation}
À titre illustratif, une fuite différentielle constante de $400 \text{ ml/min}$ durant $30 \text{ secondes}$ produit mathématiquement une accumulation de $200 \text{ ml}$.

\begin{table}[h]
\centering
\caption{\textcolor{bleuprincipal}{Récapitulatif des seuils de détection}}
\begin{tabular}{@{}lll@{}}
\toprule
\textbf{Seuil (ml/min)} & \textbf{Classification} & \textbf{Action} \\
\midrule
$\leq 300$ & Normal & Reset volume \\
$301 - 800$ & Fuite mineure & Surveillance \\
$> 800$ & Fuite critique & Alerte immédiate \\
\bottomrule
\end{tabular}
\end{table}

\section{Interface et Restitution (Frontend)}
Le Dashboard récupère l'historique traité par le serveur de façon cyclique. Les volumes et les débits sont affichés sous format numérique, vectoriel (graphiques \textsf{Recharts}) et tabulaire avec application automatique d'un formatage d'unités (bascule automatique en Litres si la valeur excède $1000 \text{ ml}$).

\section{Conclusion}
Par l'utilisation croisée du calcul d'intégration temporelle et de seuils de silence stricts, le système \textcolor{bleuprincipal}{AquaGuard} s'assure d'une haute fiabilité prédictive tout en minimisant les fausses alertes liées aux bruits de mesure conventionnels.

\end{document}